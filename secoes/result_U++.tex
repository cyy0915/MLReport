%\section{Result}
\subsection{UNet++}
We trained UNet++ with the designated parameters, along with deep supervision and without supervision, which results in the accuracies in Table.\ref{tab:unetppResult}.

\begin{table}[!htbp]
\centering
\caption{UNet++ results}\label{tab:unetppResult}
\begin{tabular}{c|c|c|c|c}
\hline
& Out 1&Out 2&Out 3&Out 4\\    
\hline
with deep supervision & 0.878 & 0.886 & 0.891 & 0.894\\
without deep supervision & / & / & / & 0.917\\
\hline
\end{tabular}
\end{table}

As shown in Table.\ref{tab:unetppResult}, deep supervised network performance increases with output layer, while still performing worse than the network without deep supervision, where in the UNet++ paper\cite{unet_pp}, the gap was not that big, and deep-supervised network perform generally better than normal ones. We suppose that was caused by our non-ideal training parameter setting. We tend to think that deep-supervised networks actually converge slower in this setting because of two reasons:
\begin{itemize}
    \item Non-decaying deep supervision force the network to figure out a good approximation right at the beginning of the network, which may over-exploit the potential of the first layer, leaving less work to do for the following layers, and making the network less competent as a whole. We argue that this setup actually slows down the training process, which will be shown in the other experiments.
    \item Small training epochs lead to early exit and non-optimal performance during training.
\end{itemize}
