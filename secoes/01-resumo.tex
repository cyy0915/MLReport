\begin{abstract}
    Medical image segmentation has always been a difficult task in computer vision, due to lack of data, hish demand on image details and huge size of input images. In this project, we adopt 3 different networks from the U-Net school to solve this problem: U-Net, U-Net++ and U-Net+++. Several data augmentation methods including rotation, shifting and clipping are used, to boost the performance of model. We conduct extensive experiment to select the optimal binarization threshold and other hyperparameters, and finally compare the pros and cons of the 3 different structures.
  \end{abstract}

  \begin{IEEEkeywords}
    U-Net, Medical Image Segmentation, Data Augmentation
  \end{IEEEkeywords}
